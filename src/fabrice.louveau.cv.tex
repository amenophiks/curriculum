\documentclass[10pt,a4paper]{moderncv}       % Only accpets 10pt, 11pt, 12pt
\moderncvtheme[blue]{classic}                % optional argument are 'blue' (default), 'orange', 'red', 'green', 'grey' and 'roman' (for roman fonts, instead of sans serif fonts)
% \moderncvtheme[blue]{casual}                  % idem

%%%%%%%%%%%%%%%%%%%%%%%%%%%%%%%%%%%%%%%%%%%%%%%%%%%%%%%%%%%%%%%%%%%%%%%%%%%%
% PACKAGES
%%%%%%%%%%%%%%%%%%%%%%%%%%%%%%%%%%%%%%%%%%%%%%%%%%%%%%%%%%%%%%%%%%%%%%%%%%%%

\usepackage[utf8]{inputenc}
\usepackage[scale=0.9]{geometry}

%%%%%%%%%%%%%%%%%%%%%%%%%%%%%%%%%%%%%%%%%%%%%%%%%%%%%%%%%%%%%%%%%%%%%%%%%%%%
% COMMANDS
%%%%%%%%%%%%%%%%%%%%%%%%%%%%%%%%%%%%%%%%%%%%%%%%%%%%%%%%%%%%%%%%%%%%%%%%%%%%

\renewcommand{\rmdefault}{phv} % Arial
\renewcommand{\sfdefault}{phv} % Arial

\renewcommand{\labelitemi}{$\triangleright$} % list icon as triangle

%%%%%%%%%%%%%%%%%%%%%%%%%%%%%%%%%%%%%%%%%%%%%%%%%%%%%%%%%%%%%%%%%%%%%%%%%%%%
% PERSONAL DATA
%%%%%%%%%%%%%%%%%%%%%%%%%%%%%%%%%%%%%%%%%%%%%%%%%%%%%%%%%%%%%%%%%%%%%%%%%%%%
\firstname{Fabrice}
\familyname{LOUVEAU}
\title{Architecte - Développeur}              
\address{23 b rue Jean Marin Naudin}{92220 BAGNEUX}
\mobile{06.08.61.29.84}                   % optional, remove the line if not wanted
% \phone{}                    % optional, remove the line if not wanted
% \fax{}                      % optional, remove the line if not wanted
\email{fabrice.louveau@gmail.com}               % optional, remove the line if not wanted
\homepage{www.doyoubuzz.com/louveau-fabrice} % optional, remove the line if not wanted 
%\extrainfo{Permis B}    % optional, remove the line if not wanted
% \photo[64pt]{picture}                       % '64pt' is the height the picture must be resized to and 'picture' is the
                                            % name of the picture file; optional, remove the line if not wanted
\quote{10 ans dans le développement de logiciels embarqués}               %
% optional, remove the line if not wanted

\nopagenumbers{}                         % uncomment to suppress automatic page numbering for CVs longer than one page

%%%%%%%%%%%%%%%%%%%%%%%%%%%%%%%%%%%%%%%%%%%%%%%%%%%%%%%%%%%%%%%%%%%%%%%%%%%%
% CONTENT
%%%%%%%%%%%%%%%%%%%%%%%%%%%%%%%%%%%%%%%%%%%%%%%%%%%%%%%%%%%%%%%%%%%%%%%%%%%%
\begin{document}
\maketitle

\section{Compétences}
\cvcomputer{Langage}{C/C++, POSIX, Python}{G-conf}{git, svn, Perforce,
ClearCase, CMSynergy}
\cvcomputer{Système}{Yocto, Buildroot, Gnome Lib, dbus}{Méthodologies}{Agile, Scrum, DO-178B}
\cvcomputer{Traçabilité}{DOORS, Reqtify}{Outils}{Eclipse, GDB, GCC,
valgrind}
\cvcomputer{Langue}{Anglais technique}{Autre}{Doxygen}

\section{Expériences}
%%%%%%%%%%%%%%
%%% Parrot %%%
%%%%%%%%%%%%%%
\cventry
{depuis\\04/2016}   % Année
{Développeur}       % Titre part 1
{Altim pour Parrot} % Titre part 2 - Employeur
{Paris}             % Titre part 3 - Ville
{}                  % Titre part 4 - 
{
  [Automotive] Intégration des fonctions Android Auto et
  Carplay dans les équipements OEM de Parrot. J'accompagne les autorités de
  certification et les équipes produits.
  \newline
  [Drone] Comparaison des encodeurs H264 matériels (Nvidia, Qualcomm) et
  logiciels
  \newline
}
%%%%%%%%%%%%%%%%%%
%%% Altim Labs %%%
%%%%%%%%%%%%%%%%%%
\cventry
{11/2015\\à 03/2016}   % Année
{Référent Technique}   % Titre part 1
{Altim}                % Titre part 2 - Employeur
{Boulogne Billancourt} % Titre part 3 - Ville
{}                     % Titre part 4 - 
{
\begin{itemize}
  \item Participation aux phases d'avant avec de potentiel client
  \item Réalisation d'un POC de streaming H264 via RTSP sur la base d'une carte
  imx6q.
  \item Mise en place d'outil interne de communication et de gestion de projet
  pour l'équipe forfait de la société.
\end{itemize}
}
%%%%%%%%%%%%%%%%%%
%%% Softathome %%%
%%%%%%%%%%%%%%%%%%
\cventry
{03/2014\\à 10/2015}       % Année
{Architecte - Développeur} % Titre part 1
{Altim pour Softathome}    % Titre part 2 - Employeur
{Nanterre}                 % Titre part 3 - Ville
{}                         % Titre part 4 - 
{
Développement d’une fonctionnalité d’aggrégation de contenus pour Livebox 4 :
\begin{itemize}
  \item  Intégration de la brique opensource Gnome Tracker (ontologie, base de
données)
  \item Développement de plusieurs briques logicielles et librairies en C
 (statistiques, manager, indexeur, torrent)
  \item Utilisation du framework Gnome Library, D-BUS, bus propriétaire
\end{itemize}
\newline{}}
%%%%%%%%%%%%%%%%
%%% Bouygues %%%
%%%%%%%%%%%%%%%%
\cventry
{09/2012\\à 02/2014}          % Année
{Architecte - Développeur}    % Titre part 1
{Altim pour Bouygues Telecom} % Titre part 2 - Employeur
{Vélizy}                      % Titre part 3 - Ville
{}                            % Titre part 4 - 
{Projet BBOX sensation (ADSL, Fibre et Cable)
\begin{itemize}
  \item Analyse de l’architecture middleware et proposition d’amélioration(process, threads, mémoire)
  \item Analyse de l’usage mémoire (fuite, sysstat)
  \item Analyse d’impact sur le système de solution technique : swap, compression kernel, rootfs
  \item Intégration logicielle orientée performance et endurance.
  \item Mise en place de KPI. Cohérence développement/validation/marketing.
  \item Recherche d’optimisation du build system (Buildroot)
\end{itemize}
\newline{}}
%%%%%%%%%%%%%%
%%% Zodiac %%%
%%%%%%%%%%%%%%
\cventry
{11/2011\\à 08/2012}   % Année
{Architecte - Zodiac}  % Titre part 1
{Altim pour Sofathome} % Titre part 2 - Employeur
{Elancourt}            % Titre part 3 - Ville
{}                     % Titre part 4 - 
{
\begin{itemize}
  \item [A400M] unité de contrôle du jaugeage :
  \item Mise au point des logiciels pour la mise en service de l’avion
  \item Gestion du standard 19 pour le logiciel opérationnel (documentation, certification)
  \item Gestion du standard 10 pour le boot loader
  \item Intervention sur toutes les étapes du cycle en V. Logiciel niveau A
  \item [Sukhoi SSJ100] calculateur de gestion de carburant :
  \item Certification du standard 9 du logiciel auprès de l’EASA
  \item Mise en oeuvre d’un nouveau profil des réservoirs de carburant (planning, développement).
  \item Intervention sur toutes les étapes du cycle en V. Logiciel niveau B.
\end{itemize}
\newline{}}
%%%%%%%%%%%%%%
%%% Safran %%%
%%%%%%%%%%%%%%
\cventry
{09/2009\\à 10/2011}       % Année
{Architecte - Développeur} % Titre part 1
{Altim pour Safran}        % Titre part 2 - Employeur
{Massy}                    % Titre part 3 - Ville
{}                         % Titre part 4 - 
{
\begin{itemize}
  \item Projet : Data Loader and Configuration System (DLCS) pour A350 XWB
  \item Rôle  de consultant sur les choix des outils de gestion de configuration, de traçabilité et de développement.
  \item Mise au point d’un package Eclipse CDT.
  \item Intégration des outils nécessaires au bon développement sous LINUX (GCC/PikeOS)
  \item Intégration sous PikeOS des modules logiciels (25 000 lignes) afin d'obtenir un exécutable.
  \item Passage de tests d'intégration bas niveau.
  \item Interlocuteur privilégié avec les équipes de validation pour l'analyse des anomalies et gestion des reprises.
  \item Projet : Software Pin Programming (SPP-BA) pour A350 XWBRédaction de la
spécifiction logicielle
  \item Mise en place d'un environnement de développement pour la plateforme SYSGO (simulateur et cible)
  \item Validation du design au travers d'une maquette, rédaction du document de design
  \item Réalisation du logiciel selon la norme DO-178B niveau C
  \item Codage et mise au point de l’application en C POSIX
  \item Mise en place de l’outil de traçabilité (Reqtify)
  \item 
\end{itemize}
\newline{}}
%%%%%%%%%%%%%%
%%% Thales %%%
%%%%%%%%%%%%%%
\cventry
{01/2007\\à 08/2009}       % Année
{Architecte - Développeur} % Titre part 1
{Altim pour Thales}        % Titre part 2 - Employeur
{Meudon}                   % Titre part 3 - Ville
{}                         % Titre part 4 - 
{
\begin{itemize}
  \item Projet : Calculateur de commande de vol (P20) pour Gulfstream G650
  \item Calculateur développé par dissemblance (2 hardware, 2 langages soit 4 configurations à gérer)
  \item Rédaction de spécification pour les drivers A429 (DO178B niveau A) et Ethernet (DO178B niveau C)
  \item Rédaction de la spécification pour l'application de qualification du calculateur (DO178B niveau D)
  \item Pilotage en off-shore d'une équipe de développement de l'application de qualification en Roumanie
  \item Pilotage de l'équipe de test de l'application de qualification en France
  \item Mise au point de l'application de qualification sur environnement simulé
  \item Intégration système de l'application de qualifiction avec les drivers sur cible (mono carte)
  \item Intégration système de l'application de qualification du calculateur (4 cartes)
  \item Projet : Système FADEC pour turbine d'hélicoptère
  \item Mise à jour des documents de spécification pour le driver A429
  \item Conception et développement du logiciel pour le driver A429
  \item Reprise de la conception et du développement du driver CAN suite aux évolutions des besoins
  \item Intégration sur banc logiciel
  \item Intégration système : BOOT, MiddleWare, drivers et application de tests
  \item Mise au point du logiciel applicatif de qualification
\end{itemize}
\newline{}}
%%%%%%%%%%%%%%%%%%
%%% Elios Info %%%
%%%%%%%%%%%%%%%%%%
\cventry
{09/2005\\à 12/2006}    % Année
{Testeur - Développeur} % Titre part 1
{Elios Informatique}    % Titre part 2 - Employeur
{Région parisienne}     % Titre part 3 - Ville
{}                      % Titre part 4 - 
{
[Projet M51]: Développement ADA et tests sur missile nucléaire
\newline{}
[Projet SCI, A380] Suivi des sous-traitant indiens pour les tests unitaires
\newline{}
[Projet ETRAC, A380] Rédaction plan d’intégration/acceptation et passage
  des test
\newline{}}


\section{Diplômes et Études}
\cventry{2003}{Ingénieur en Elecronique et Informatique}{Université d'Orléans}{(45)}{}{Spécialité Systèmes Embarqués}
\cventry{1999}{D.U.T Mesures Physiques}{Université d'Orsay}{(91)}{}{Spécialité Techniques Instrumentales}
\cventry{1997}{Baccalauréat Science et Technique de Laboratoire}{Lycée Fresnel}{(75)}{}{Spécialité Optique}

\section{Centres d'intérêt}
\cvcomputer{}{Maquette (voiture, aéronefs)}{}{Jeux de plateau/carte}

\end{document}\grid